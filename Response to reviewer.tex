\documentclass[12pt,twoside,reqno]{amsart}
%\documentclass[12pt]{article}

\usepackage{xcolor}
\usepackage{hyperref}
\usepackage{amsthm, amsmath, amscd, amssymb,centernot}
\usepackage[all]{xy}
\usepackage[T1]{fontenc}
\usepackage[left=2.5cm,top=2.5cm,bottom=3cm,right=2.5cm]{geometry}
\usepackage{amsthm, amsmath, amscd, amssymb,centernot}
\usepackage{comment}
\setlength{\headheight}{15.2pt}
\setlength\parskip{.1in}
\setlength\parindent{0.2in}

%\usepackage{tikz}https://www.overleaf.com/project/642e283246ec4730b8bdf398
\usepackage{csquotes}

\newcommand{\s}{\varepsilon}

\numberwithin{equation}{section}
\newtheorem{theorem}{Theorem}[section]
\newtheorem*{theorem*}{Theorem}
\newtheorem{proposition}[theorem]{Proposition}
\newtheorem{lemma}[theorem]{Lemma}
\newtheorem{corollary}[theorem]{Corollary}
\newtheorem{conjecture}[theorem]{Conjecture}
\newtheorem{definition}[theorem]{Definition}
\theoremstyle{definition}
\newtheorem{question}[theorem]{Question}



\begin{document}
\title{Response to Reviewer}
\begin{description}
\item [Mauscript Number] IJM-D-22-00005 \\
\item [Full Title] Seshadri constants on some blow ups of Projective spaces
\end{description}

We start by thanking the anonymous referee for carefully reading the manuscript and suggesting 
useful changes and also for pointing out an error in the definition of ``\textit{multiplicity of a curve along a line in $\mathbb{P}^3$}''. We hope that after making these changes, the paper has becomes more accessible for the readers and its overall quality has improved considerably. We have made all the changes suggested by the referee and some other modifications, which we felt necessary. Following are the detailed (point by point) explanation of changes that we have made on the basis of the report. \\

We agree with the referee about the problem in our way of using the definition of multiplicity of an irreducible curve $C$ along a line $l$, both 
contained in $\mathbb{P}^3$. We have made it clear in the manuscript now that this is defined as the length of zero dimensional 
scheme $C \cap l \subset \mathbb{P}^3$, as pointed out by the referee. On the basis of this new (correct) definition, we have modified our arguments in theorems of Section $4$ and that of Section $5$. We now get some restricted results about Seshadri constants in case of $r=5,6$ line blow ups of $\mathbb{P}^3$ (Proposition 4.3), $r=4,5,6$ and $7$ line blow ups of $\mathbb{P}^4$ (Proposition 5.4) and $r=3,4$ and $5$ line blow ups of $\mathbb{P}^5$ (Remark 4).  In the cases not covered in the above list, we get bounds mainly. We give precise values of Seshadri constants only when points are special, for example (2), (3) and (4) of Theorem 4.2, (2), (3) of Theorem 5.3, and (2) of Theorem 5.5. However we do compute $\varepsilon(X_{r,0}^k,L)$ in various cases which appear in Remarks 1,2, and 5 and are explained towards the end of this letter.\\

\textbf{List of minor mathematical issues we fixed:}
\begin{enumerate}
 \item We have changed the arguments in page 3, line -2, as per referee's suggestions.  
 \item We have mentioned now in line -9 of page 4 that we use (2) to conclude the result. 
 \item We have set $m_i = 0$ for $i = s+1,..., 2n$ and moved $m_i + m_{i+n}$ after the summation on page 5, line 1-2. 
 \item We deleted the second $\sum\limits_{i=1}^s$ in the repetition. 
 \item We have added the missing case in the proof of Theorem 3.3 on page 5, line 12. 
 \item On page 6 line -2, we have added a precise reference namely ``\textit{from inequality (3) of Theorem 3.3}''. 
 \item We replaced $d>0$ with $d \geq 0$ in the statements of Theorem 4.1 (1), (2), (3) and also in the proof. 
 \item We replaced $d>0$ with $d \geq 0$ in the statements of Theorem 5.1 (1), (2), (3), Proposition 5.2 and also in the proofs. 
\end{enumerate}

\textbf{List of grammar issues/typos we fixed:}
\begin{enumerate}
 \item Rectified the spelling mistake in Demailly. 
 \item ``One of the very important'' has been changed to ``One of the most important'' page 1, line -11.
 \item We changed ``Seshadri constant can be'' to ``Seshadri constants can be'' on page 1, line -6.
 \item Done.
 \item We replaced the phrase ``d is large enough as in Proposition 3.2'' with ``d is large enough namely 
 $\sum_{i=1}^s m_i - nd \leq b_l$, where $b_l$ is as in Proposition 3.2''.
 \item We changed the notation $H_2$ by $H'$ on page 6, line -9. 
 \item Done.
 \item Does not exist in the modified version.
 \item Does not exist in the modified version.
 \item Does not exist in the modified version.
\end{enumerate}

\textbf{Some other small changes which we have done:} \\
Second part of the paper is rewritten to some extent. We give here the detailed account of changes that we have done from the previous version.
\begin{enumerate}
    \item We introduce some notations in the beginning of \textit{Section 4}, which we use in the proof of Theorem 4.2 and in later part of the paper. We have also added a paragraph before Theorem 4.2 where we explain ``the meaning of multiplicity of $C$ along a line $l$''. 
    \item After Theorem 4.2, we have added a \textit{Remark 1} where we compute $\varepsilon(X_{r,0}^3,L)$ for $r=1,...,4$.
    \item We then added a Proposition 4.3 giving some partial results about $\varepsilon(X_{r,0}^3,L)$ for $r=5$ and 
    a \textit{Remark 2} bounding $\varepsilon(X_{5,0}^3,L)$. We leave $r=6,7$ cases as the results in those cases are similar to Prop. 4.3 and Rem. 2.
    \item In section 5, we have added a paragraph before Theorem 5.3, where we introduce some notation about the intersection of joins of lines $J(l_i, l_j)$ and of $J(l_i, l_k)$.
    \item After Theorem 5.3, we added a \textit{Remark 3}, where we compute $\varepsilon(X_{4,0}^4,L,x)$ when $x \in l_j \cap J(l_i,l_k)$.
    \item We then added a Propostion 5.4, where we get partial result about $\varepsilon(X_{4,0}^4,L)$ (we again leave $r=5,6,7$ cases).
    \item We also note in \textit{Remark 4}, that similar results like Prop. 5.4 holds for $\varepsilon(X_{r,0}^5,L)$ 
    for $r=3,4,$ and $5$ (note that for $r=1,2$ we have Theorem 5.5).
    \item Finally in \textit{Remark 5}, we have added a note about the global Seshadri constant $\varepsilon(X_{r,0}^k,L)$ for $k=4,5$ and $r=1,...,7$, $r=1,...,5$ respectively.
\end{enumerate}

\end{document}