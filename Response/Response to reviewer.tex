\documentclass[12pt,twoside,reqno]{amsart}
%\documentclass[12pt]{article}

\usepackage{xcolor}
\usepackage{hyperref}
\usepackage{amsthm, amsmath, amscd, amssymb,centernot}
\usepackage[all]{xy}
\usepackage[T1]{fontenc}
\usepackage[left=2.5cm,top=2.5cm,bottom=3cm,right=2.5cm]{geometry}
\usepackage{amsthm, amsmath, amscd, amssymb,centernot}
\usepackage{comment}
\setlength{\headheight}{15.2pt}
\setlength\parskip{.1in}
\setlength\parindent{0.2in}

%\usepackage{tikz}https://www.overleaf.com/project/642e283246ec4730b8bdf398
\usepackage{csquotes}

\newcommand{\s}{\varepsilon}

\numberwithin{equation}{section}
\newtheorem{theorem}{Theorem}[section]
\newtheorem*{theorem*}{Theorem}
\newtheorem{proposition}[theorem]{Proposition}
\newtheorem{lemma}[theorem]{Lemma}
\newtheorem{corollary}[theorem]{Corollary}
\newtheorem{conjecture}[theorem]{Conjecture}
\newtheorem{definition}[theorem]{Definition}
\theoremstyle{definition}
\newtheorem{question}[theorem]{Question}



\begin{document}
\title{Response to Reviewer}
\begin{description}
\item [Mauscript Number] GMJ 231005 \\
\item [Full Title] Seshadri Constants of Curve Configurations on Surfaces
\item [Author] Krishna Hanumanthu, Praveen Roy and Aditya Subramaniam
\end{description}

We start by thanking the anonymous referee for carefully reading the manuscript and suggesting 
useful changes and also for pointing out an error in the Assumption 2.1 which lead to the ambiguity in Theorem 2.5. We also agree with the reviewer that the example 2.21 is 
better to be shorten focusing only on the computation related to Seshadri constants. We have made all these changes, and we hope that after making these changes, the paper has becomes more accessible for the readers and its overall quality has improved considerably.  Following are somewhat detailed explanation of changes that we have made on the basis of the report. \\

We agree with the referee about the problem in our way of using the Assumption 2.1. While writing the paper, we always had in mind the \textbf{connectedness of the arrangement}, 
which we somehow missed to incorporate in the Assumption 2.1. We have made it clear in the manuscript now. Note that, now we don't require the semi-ampleness of divisor $A$ for $C_i^2$ to be positive, and it follows 
from the connectedness of the arrangement. Additionally, we would prefer to retain the second part of Assumption 2.1, as removing that requires significant amount of additional work which we are currently not inclined to pursue.
 Regarding the reviewer's final concern, we have revised Example 2.21, shortening it to focus solely on the computation of the Seshadri constant. \\

We once again thank the referee for a quick review, we hope that we addressed all the concern of the referee regarding this manuscript.


\end{document}