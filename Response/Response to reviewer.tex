\documentclass[12pt,twoside,reqno]{amsart}
%\documentclass[12pt]{article}

\usepackage{xcolor}
\usepackage{hyperref}
\usepackage{amsthm, amsmath, amscd, amssymb,centernot}
\usepackage[all]{xy}
\usepackage[T1]{fontenc}
\usepackage[left=2.5cm,top=2.5cm,bottom=3cm,right=2.5cm]{geometry}
\usepackage{amsthm, amsmath, amscd, amssymb,centernot}
\usepackage{comment}
\setlength{\headheight}{15.2pt}
\setlength\parskip{.1in}
\setlength\parindent{0.2in}

%\usepackage{tikz}https://www.overleaf.com/project/642e283246ec4730b8bdf398
\usepackage{csquotes}

\newcommand{\s}{\varepsilon}

\numberwithin{equation}{section}
\newtheorem{theorem}{Theorem}[section]
\newtheorem*{theorem*}{Theorem}
\newtheorem{proposition}[theorem]{Proposition}
\newtheorem{lemma}[theorem]{Lemma}
\newtheorem{corollary}[theorem]{Corollary}
\newtheorem{conjecture}[theorem]{Conjecture}
\newtheorem{definition}[theorem]{Definition}
\theoremstyle{definition}
\newtheorem{question}[theorem]{Question}



\begin{document}
\title{Response to Reviewer}
\begin{description}
\item [Manuscript Number] GMJ 231005 \\
\item [Title] Seshadri Constants of Curve Configurations on Surfaces
\item [Authors] Krishna Hanumanthu, Praveen Roy and Aditya Subramaniam
\end{description}

We thank the referee for carefully reading the manuscript and suggesting 
many useful changes and also for pointing out an error in the Assumption 2.1 which leads to 
ambiguities at several places. We fixed this error by assuming that the arrangement is connected. 

We also agree with the referee that Example 2.21 can be shortened and it is better to 
focus on the computation of Seshadri constants. 

Below we describe in detail the changes made to the manuscript based on the referee's comments. 
%We have made all these changes, and we hope that after making these changes, the paper has becomes more accessible for the readers and its overall quality has improved considerably.  Following are somewhat detailed explanation of changes that we have made on the basis of the report. \\
\begin{itemize}
\item Following the referee's suggestion, we included the assumption that the arrangement $\mathcal{C}$ is \textit{connected}. We added a remark immediately after Assumption 2.1 mentioning that the connectedness assumption implies that $C_i \cdot C_j > 0$ for all $i,j$.

\item We also removed the unnecessary assumption that all the curves do not meet at any point. Referee correctly noted that this is not necessary for the main results. 

We only needed this assumption while computing the configurational Seshadri constants in Section 2.3. We therefore added the required assumption at the beginning of Section 2.3. 

%agree with the referee about the problem in our way of using the Assumption 2.1. While writing the paper, we always had in mind the \textbf{connectedness of the arrangement}, 
%which we somehow missed to incorporate in the Assumption 2.1. We have now made it clear in the manuscript. \\
%\item Note that, now we don't require the semi-ampleness of divisor $A$ for $C_i^2$ to be positive, and it follows 
%from the connectedness of the arrangement. 
\item We retained the condition that all curves in our arrangements are linearly equivalent to a specific divisor. Many of the results and questions we state depend on this assumption. So removing this assumption will require a substantial re-working of the paper. 

%Additionally, we would prefer to retain the second part of Assumption 2.1, as removing that requires 
%significant amount of additional work which we are currently not inclined to pursue. In-fact even the statement of Theorem 2.5 will change, if we remove the second assumption. 
%By the phrase `curves are linearly equivalent to a fixed divisor $A$', we mean that, $C_i \in |A|$ (linear system of $A$) for all $i$. \\

\item We have revised Example 2.21 considerably. We shortened it to focus solely on the computation of the Seshadri constant.
\end{itemize}


%We once again thank the referee for a quick review, we hope that we addressed all the concerns of the referee regarding this manuscript.


\end{document}